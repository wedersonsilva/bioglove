\documentclass{../ufpatcc}
\usepackage{lipsum}
\begin{document}
% FRONT MATTER -------------------------------------------%
  \capa
  \folhaderosto
  \pagenumbering{arabic}
  \fichacatalografica{Biblioteca Prof. Dr. Clodoaldo Beckmann}
                     {1. Área 2. Outra área I. Título.}
                     {21.ed. 001.1}
  \folhadeaprovacao{17/06/2018}
  \begin{dedicatoria}
    \lipsum[13]
  \end{dedicatoria}
  \begin{agradecimentos}
    \lipsum[1-5]
  \end{agradecimentos}
  \begin{epigrafe}
    ``\lipsum*[101]''\par
      Autor
  \end{epigrafe}
  \begin{resumo}
    \lipsum[1]
  \end{resumo}
  \begin{abstract}
    \lipsum[1]
  \end{abstract}
  \tableofcontents

	% --------------------- 
	%	INICIO DOS CAPÍTULOS	
	% --------------------- 
	
	% --------------------- 
	%	INTRODUÇÃO 	
	% --------------------- 
  \chapter{Introdução} % ou \input{arquivoexterno}
		
		Contextualização; Estado da arte (se tiver); Motivação; O que vai fazer; Metodologia; O que terá no resto do documento;


	% --------------------- 
	%	REFERENCIAL TEÓRICO	
	% --------------------- 
	\chapter{Referencial Teórico}

		\section{Sensor Flex}

		\section{Potenciômetro}
		Segundo a lei de Ohm, dada uma corrente constante, ao variar a resistência teremos uma variação da tensão ($V = R.I$). O potenciômetro é um componente eletrônico que permite, através do giro do seu eixo, a variação da resistência entre seus terminais. Dentre as principais características de um potenciômetro estão o valor máximo de sua resistência, seu número de voltas, seu grau máximo de giro (aproximado) e se ele é liner ou logarítmico.

		Sendo assim, para uma corrente constante, ao girar o eixo do potenciômetro, dependendo do sentido do giro, perceberemos um aumento ou diminuição da tensão naquele ponto. Partindo de um ponto extremo com resistência mínima até o outro ponto extremo no qual a resistência deverá ser a máxima característica do componente.

		\section{Movimentação dos dedos}

		\section{Arduíno}


	% ----------------------------
	%	TRABALHO PROPRIAMENTE DITO	
	% ----------------------------
	\chapter{Trabalho Propriamente Dito}
	
		\section{Medidas e Posicionamento}

		\section{Placa de Circuito Impresso}
		
		\section{Movimento mecânico}


	% ----------------------------
	%	RESULTADOS E ANÁLISES	
	% ----------------------------
	\chapter{Análises e Resultados}

		\section{Configurações}

		\section{Testes}

		\section{Resultados}
		

	% ----------
	%	CONCLUSÃO	
	% ----------
  \chapter{Conclusão}

		\section{Conclusões}

		\section{Trabalhos Futuros}

	% ----------------------------
	%	REVISÃO BIBLIOGRÁFICA	
	% ----------------------------
  \chapter{Revisão Bibliográfica}
    \lipsum[1]\cite{atalholivro}
    \lipsum[2]\cite{atalhoonline}


% BACK MATTER --------------------------------------------%
  \appendix
    \chapter{Algum apêndice}
    \chapter{Outro apêndice}

  \bibliographystyle{IEEEtran}
  \referencias{referencias}
\end{document}
